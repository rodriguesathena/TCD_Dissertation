\section*{Literature Review}
\addcontentsline{toc}{section}{Literature Review}

    Framing transit reliability in high-frequency systems requires both conceptual insight and methodological adaptability. This study builds on established strands of research, drawing from prior work on behavioural metrics, the use of real-time prediction data as a proxy for operational performance, and evolving understandings of how public transport systems responded to the pressures of the COVID-19 pandemic. Together, these foundations shape both the methodological choices and the interpretive lens through which the LUAS is assessed.

\phantomsection
\subsection*{Operational Metrics for Transit Service Reliability}
\addcontentsline{toc}{subsection}{Operational Metrics for Transit Service Reliability}

    Service reliability in public transportation refers not only to a vehicle’s ability to arrive ‘on time’ according to schedule but, more importantly, to the system’s ability to maintain consistent and predictable activity. Typically assessed using performance indicators that reflect operations and passenger experience,  this review focuses on three core metrics for analysis: headway regularity, travel time volatility, and journey duration. Each of these captures distinct but interconnected aspects of reliability, which are widely cited as essential for evaluating the transit efficacy.

    Consistently highlighted as a crucial factor in influencing overall performance and reliability in transit research, headways are the time interval between two consecutive vehicles operating on the same route in the same direction \parencite{abkowitz1978transit}. In high-frequency transit systems, where passengers rely less on scheduled timetables and more on the expectation of consistent vehicle spacing, headway regularity offers a more meaningful measure of service quality than traditional punctuality methods. Unlike schedule adherence, which compares vehicle arrivals against a fixed timetable, headway regularity reflects how evenly spaced vehicles are over time, directly influencing passenger wait times and system fluidity \parencite{albright2012transit}.  Prior research supports the use of inferred headways from AVLS data as a reliable foundation for evaluating performance in such contexts. For example, \textcite{furth2007optimal} argue that metrics such as headway deviation and excess waiting time offer a more accurate reflection of passenger experience than on-time statistics. This approach is further supported by \textcite{ma2014measuring}, who highlight the value of AVLS for identifying and correcting irregular service patterns in real time. As such, headway regularity is incorporated as a prime indicator to assess consistent spacing across the LUAS network.

    While headway regularity captures vehicle spacing, it does not account for variation in trip length. Accordingly, \textcite{vanoort2011service} emphasises the importance of capturing multiple dimensions in quick-turnaround operations, especially relating to variability in both arrival patterns and travel timings. To capture this different, yet equally important dimension of service reliability, travel time volatility (TTV) measures the inconsistency in the duration it takes for a transit vehicle to complete a given route, encompassing operational instabilities from unobserved factors such as congestion, signal delays, and varying dwell times. Scholars argue that even these minor unpredictabilities can undermine trustworthiness and reduce passenger confidence, especially in transit where frequency replaces stretch scheduling as the basis for reliability, as seen in the LUAS system \parencite{tirachini2022headway}. From a rider’s perspective, predictability is often more important than strict adherence to timetables, with studies consistently showing that passengers are more sensitive to TTV on arrival than to minor delays. \textcite{benezech2013value} describe this effect as a ‘hidden waiting time’ where unpredictable journey durations create a perception of lost time, even if vehicles are not technically late in reaching the station. Their findings suggest that consistency is often valued more highly than speed in frequent-route networks. Similarly, \textcite{diab2012understanding} demonstrate that even slight improvements in perceived regularity can result in significant gains in rider satisfaction. Accordingly, this study uses travel time volatility to capture short-term fluctuations in duration and service inconsistencies that traditional schedule-based metrics may overlook. 

    Journey duration offers a system-level perspective on reliability not captured by headway regularity or TTV,  that remains meaningful even in the context of segmented passenger journeys. As a key indicator, journey duration refers to the total time required for a vehicle to complete a full trip from origin to destination. It can be evaluated through various methods, including average travel times as well as buffer and planning indices, which aim to capture the contingency passengers required to be considered ‘on time’ \parencite{ma2014measuring}. Other studies have used AVLS to reconstruct full trips and assess reliability by combining in-vehicle travel duration with waiting time, as shown by \textcite{zhang2022prediction} in their evaluation of end-to-end trip consistency. Collectively, this literature supports the inclusion of journey duration as a critical complement to spacing- and variability-based metrics, offering a fuller representation of passenger experience across entire routes.

    These three measures form the analytical backbone of this study, providing a multidimensional framework through which LUAS reliability is assessed from a technical and user experience standpoint. They allow for the identification of spatial and temporal patterns where service performance may deviate from expectations, aligning with passenger perceptions of unreliability. As proxies for quality, they support a behaviourally informed approach to understanding how operational irregularities are experienced at ground level.

\phantomsection
\subsection*{Forecast Data as a Proxy for Observed Transit Performance}
\addcontentsline{toc}{subsection}{Forecast Data as a Proxy for Observed Transit Performance}

    A common limitation in evaluating transit is the lack of access to continuous, stop-level records of actual vehicle arrivals. In contexts where observed arrival data is unavailable, predicted arrival times derived from real-time feeds offer a viable alternative for approximating vehicle behaviour. Inferred changes and disappearances in forecast data, such as fluctuations in predicted arrival times or removal of upcoming trams, have been used in transit research to approximate real tram events like arrivals and departures. Treated as an evolving data stream, these structured forecast patterns allow researchers to reconstruct vehicle trajectories and evaluate network performance in contexts where observed data is unavailable \parencites{sun2016smart}{muller2001trip}. These techniques establish the validity of using inferred events for performance analysis when direct observation is not possible, particularly in regular-interval services where dwell variability and small spacing shifts can disproportionately affect perceived reliability.

    These principles are applied by using time-stamped LUAS forecast snapshots, collected systematically via the public-facing Application Programming Interface (API). Although real-time predictions are published continuously, no archival mechanism exists, necessitating custom logging at consistent intervals. Forecast disappearances are then aligned with structured stop templates and filtered by line and direction to reconstruct tram journey sequences. The method enables consistent identification of stop-level events and spacing patterns without relying on direct arrival observations. \textcite{xu2017arrival} show that prediction accuracy varies systematically with time of day and network load, underscoring the importance of temporal sampling in reliability analysis. This data-driven reconstruction process aligns with broader trends in transit research that leverage real-time feeds to monitor and evaluate system performance, particularly in limited environments. Capturing the evolution of predictions across time supports the extraction of dynamic features that are difficult to observe through static scheduled data.

    Beyond the stop level, this study employs route-aligned identifiers to stitch sequences of inferred arrivals into full tram trajectories. These are then grouped and categorised by pandemic phase, period, direction, and line. This spatial extension allows for TTV and journey duration to be calculated not just at single nodes, but over entire segments or full-line runs. Segment-level analysis has been shown to more accurately reflect the experience of riders, especially in urban transit where issues may cluster in specific zones due to junctions, signal delays, or boarding variability \parencite{zuniga2021estimation}. This approach aligns with broader methodological frameworks that recommend spatially distributed inference to analyse rider exposure to disruption, headway instability, and system resilience \parencite{cats2016risk}. Importantly, these network-wide metrics can be disaggregated by peak periods or pandemic phases, enabling both station-specific and system-level assessments of the LUAS.

    Snapshot-based inference fills a methodological gap by providing a replicable and scalable means of reconstructing service activity using only forecast data. In the absence of direct logs, this approach still supports robust evaluation of core reliability metrics while reflecting a broader shift in transport research toward high-frequency, behaviourally relevant indicators grounded in AVLS data \parencite{tirachini2022headway}. By applying this technique to the LUAS, this study provides a practical and replicable methodology for evaluating system consistency under real-world constraints and offers insight into how public APIs can be utilised to build datasets where direct operational information is withheld.

    Together, these methods validate the use of forecast-based inference as a robust substitute for direct operational data in assessing reliability. By recreating tram behaviour from prediction snapshots, this study can evaluate system performance across both spatial and temporal dimensions, even in the absence of observed arrival logs. The approach not only reflects emerging practices in transit analytics but also ensures that the indicators used are grounded in observable, replicable studies, providing a scalable foundation to analyse network consistency while offering a meaningful way to assess passenger-facing performance within data-constrained environments. While elements of these techniques have been applied elsewhere, this study is among the first to adapt them specifically to Dublin’s LUAS system, demonstrating how fragmented forecast streams can be recombined to evaluate reliability in a previously unexamined context.



\phantomsection
\subsection*{Pandemic Impacts on Transit Behaviour and Reliability}
\addcontentsline{toc}{subsection}{Pandemic Impacts on Transit Behaviour and Reliability}

    The COVID-19 pandemic caused an unprecedented shock to all aspects of daily life, rapidly disrupting public transportation worldwide. Demand patterns, rider behaviour, and service frequency shifted almost overnight, placing strain on transit infrastructure that attempted to maintain operational reliability amid evolving government safety regulations. Observed through historical forecast data, this time serves as a natural stress test for the LUAS network’s adaptability and resilience, with core themes such as passenger trust and stability emerging as central in the evaluation of the impact of pandemic-induced disruption on urban mobility systems \parencite{benita2021human}.

    Traditional approaches of transit research relying on stable demand assumptions, fixed schedules, or long-term averages became inadequate for understanding system performance under these changing conditions. As a result, studies called for data-driven frameworks that could measure real-time adaptability and assess how infrastructure respond to external shocks \parencite{gutierrez2021covid}; or focused on examining how operational trade-offs during lockdowns, such as reduced service levels, affected long-term reliability and rider trust \parencite{deborger2021covid}. Researchers also turned to higher-frequency data and temporal segmentation techniques to better understand system behaviour during lockdowns and reopenings. Rather than treating the pandemic as an outlier, it was positioned as a critical opportunity to examine structural weaknesses and the limits of network flexibility. Collectively, this body of work has expanded the analytical lens through which transit performance is evaluated, encouraging more responsive and situationally-aware approaches.

    In the Irish context, COVID-19 significantly altered public travel behaviour and usage, with a steep nationwide decline in ridership that raised concerns about long-term behavioural shifts away from transit due to safety and comfort anxieties \parencite{hynes2020utility}. During partial reopening phases in Dublin, many remained hesitant to resume the use of transit such as the LUAS, especially in densely populated corridors, highlighting issues of accessibility and behavioural adaptation across different demographic and geographic contexts \parencite{caulfield2021trinity}. Meanwhile, \textcite{marra2022impact} provide longitudinal evidence of pandemic-driven changes in travel timing, route preferences, and transfer patterns, all of which suggest significant disruption to previously stable urban mobility norms. These findings contribute to a localised understanding of how the pandemic impacted operations, rider confidence, and overall transit use.

    Both international and Ireland-specific research supports the segmentation of data into pandemic-related phases for comparative analysis. These studies justify the use of historical forecast data to examine service regularity, inferred arrival patterns, and volatility under conditions of external stress. However, few have combined multiple internal reliability metrics to evaluate system performance during this period, limiting a more holistic understanding of how different dimensions of service consistency evolved under pandemic conditions. In the context of LUAS, this approach allows for a deeper understanding of how reliability shifted in response to fluctuating demand, constrained operations, and evolving public confidence—factors essential to interpreting performance in a time defined by uncertainty.

