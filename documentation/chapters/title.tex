\begin{titlepage}
    \centering
    % Image at the top
    \includegraphics[width=10cm]{figures/paper_figures/tcd-logo.png}\par
    \vspace{1cm}

    % Department and School
    {\large Department of Political Science\par}

    \vspace{2.5cm}

    % Title section
    {\LARGE\bfseries Forecasting Reliability:\par}
    \vspace{0.3cm}
    {\Large\bfseries LUAS Performance Across Crisis and Recovery\par}

    \vspace{3cm}

    % Name
    {\Large Athena Rodrigues\par}

    \vspace{1.5cm}

    % Supervisor and date
    {\large Supervisor: Dr. Jian Cao\par}
    \vspace{0.5cm}
    {\large August 2025\par}

    \vspace{2cm}

    % Submission information
    {\large Submitted in partial fulfilment of the requirements for the award of the degree\par}
    \vspace{0.3cm}
    {\large Master of Science (MSc) in Applied Social Data Science\par}
    \vspace{0.3cm}
    {\large Trinity College Dublin\par}
\end{titlepage}

\clearpage
\thispagestyle{plain}


\section*{Abstract}

This study investigates the operational reliability of the LUAS light rail network in Dublin by reconstructing tram journeys from public Automatic Vehicle Location System (AVLS) forecast data. Using a custom methodology developed for a data-constrained environment, it analyses three core performance metrics across four pandemic-defined phases from 2020 to 2022: headway regularity, travel time volatility, and journey duration. The findings indicate that while central sections of the LUAS corridors displayed relative stability, peripheral zones and outbound services suffered from extended headways, inconsistent timing, and spatial imbalance; many of which persisted even after restrictions were lifted.

Methodologically, the study demonstrates that forecast data can be systematically reconstructed to infer tram trajectories and diagnose network-level irregularities without requiring proprietary or per-unit records. These findings offer both practical and conceptual contributions: showing how open data can support evidence-based planning and illustrating the structural limitations of frequency-based service models when reliability is unevenly distributed. By mapping the patterns of dependability visible under constrained conditions, the research underscores the importance of equity, adaptability, and transparency in the evaluation of transport performance. It presents a scalable and replicable framework for identifying operational inconsistencies and calls for more responsive planning tools that integrate technical monitoring with everyday service realities.

\clearpage
\thispagestyle{plain}

\section*{Acknowledgements}

I would like to thank my supervisor, Dr. Jian Cao, for their thoughtful guidance and support throughout this project. Their feedback was invaluable in helping me refine both the scope and the clarity of my work. I also wish to acknowledge the faculty and students of the Applied Social Data Science program at Trinity College Dublin for providing a collaborative and intellectually engaging environment that shaped the development of this dissertation.


I am also grateful to my friends, both here and back home, for their encouragement and steady support over the course of this year. Finally, I would like to thank my parents for their continued belief in me and for the foundation of support that has made this work possible. Their patience and generosity throughout my academic journey have been greatlyy appreciated.

\clearpage
