\section*{Analysis}
\addcontentsline{toc}{section}{Analysis}

\phantomsection
\subsection*{Overview}
\addcontentsline{toc}{subsection}{Overview}

    While the LUAS network operates in a context where a degree of variability is expected, the benchmarks published by Transport for Ireland (TFI) provide a valuable reference point for assessing operational reliability. This analysis focuses on the previously discussed metrics as each captures a distinct dimension of operational performance and is assessed in line with the four phases, with further disaggregation by time of day to account for both structural and temporal variation in network behaviour. Headway regularity reflects the spacing of stop-level service, TTV captures the variability between stops, and journey duration provides a measure of overall efficiency across full routes. Given the influence of route geometry and stop spacing on standard deviation, visualisations help contextualise and highlight otherwise obscured patterns; in particular, graphical outputs identify spatial asymmetries, directional trends, and clusters of disruption across the network. As no observed arrival logs or vehicle-level identifiers are publicly released by Transport for Ireland, formal validation was not possible; however, internal checks were built into each processing stage to enforce strict matching conditions using stop templates and temporal thresholds. In addition, random spot checks of reconstructed journeys confirmed that inferred trajectories followed plausible timing, directionality, and stop progression. Although each metric is presented independently in the sections that follow, their interrelationships and evolution across pandemic phases are considered collectively to support a broader understanding of network stability.

\phantomsection
\subsection*{Headway Regularity}
\addcontentsline{toc}{subsection}{Headway Regularity}

    Headway regularity reflects the consistency in timing between consecutive tram arrivals at a given stop and is a core determinant of perceived reliability in high-frequency networks. While the LUAS does not operate on a fixed timetable, Transport for Ireland recommends a target headway of 3 to 5 minutes during peak hours as a benchmark for optimal service delivery  \parencite{tii_luas}. This target timing forms a baseline against which operational deviations, bunching, or long waits can be evaluated. Within a transit system like the LUAS, where schedules are implicitly embedded in passenger expectations, irregular headways can have disproportionate effects on crowding, platform dwell times, and overall journey time reliability. These patterns are illustrated in the Appendix section Headway Regularity, which visualises spatial clusters of service volatility across both time and direction. A consistent structural pattern emerges with central segments of both lines reliably maintaining moderate, predictable headways, while outer termini experienced sustained irregularity, notably during evening peaks and outbound travel.

    Morning services were generally more structured, mainly along inner corridors. Across all phases, inbound travel through central stops such as Smithfield, Four Courts, Parnell, and Marlborough maintained average headways below 9 minutes, even during the lockdown. This consistency likely reflects strategic prioritisation of commuter flows into the city core during high-demand hours, where service regularity is both more visible and more operationally consequential. In contrast, outer zones such as Cherrywood, Fortunestown, and Cabra frequently recorded headways over 20 minutes, with only marginal improvement in the post-COVID phase. This suggests enduring gaps in frequency at the edges of the network, where demand may be lower but reliability remains essential for equity of access. The persistence of this disparity across time implies that morning service recovery efforts disproportionately favoured high-demand core areas, potentially at the expense of peripheral users.

    Evening peak headways exhibited more pronounced directional imbalance, highlighted on outbound routes. This asymmetry likely reflects both higher return demand and residual disruption from earlier midday delays, which can propagate across the schedule and accumulate by late afternoon. Trams departing the city towards Saggart, Cherrywood, and Hospital consistently exhibited the longest headways, often surpassing 22 minutes. While inbound trams into the city centre remained relatively stable at 10 to 13 minutes across most phases, outbound frequencies remained fragmented even after official service recovery. This imbalance is particularly consequential during evening peaks when passengers rely on dependable return journeys to suburban or peripheral areas. It also underscores the operational challenges of maintaining bidirectional regularity across an extended network span with finite trams and staffing resources.

    Taken together, these trends reveal a resilient yet spatially uneven service model. The LUAS was able to preserve moderate headway consistency across central corridors, especially for inbound morning travel, reinforcing the network’s functional backbone. However, this stability came at the cost of persistent irregularity and extended gaps at peripheral stops, particularly during evening outbound periods. Importantly, the network’s post-pandemic recovery did little to rebalance this spatial disparity. Instead, it reinforced a hierarchy of regularity, where inner-city predictability was preserved, and outer-zone variability remained structurally embedded. For passengers, especially those living or working beyond central Dublin, this translates to longer effective wait times, increased uncertainty, and a diminished perception of service dependability. These patterns raise critical questions about the equity of resource allocation, the limits of operations in extended networks, and the degree to which transit systems can, and should, sustain balanced service trustworthiness across all user geographies.

\phantomsection
\subsection*{Travel Time Volatility}
\addcontentsline{toc}{subsection}{Travel Time Volatility}

    While headway regularity captures frequency stability at the stop level, travel time volatility reveals the extent to which journey consistency fluctuates along the network’s spatial and directional axes. Travel time volatility (TTV) refers to the variability in travel durations between consecutive tram stops. This is averaged across multiple trips within the various defined timeframes and directions. Maintaining low TTV is essential to sustaining perceived reliability for rapid transit systems. While the TFI does not specify an official threshold, this analysis adopts a one-minute benchmark, indicating that in a system where typical rider trips span short segments, average deviations above one minute reflect meaningful inconsistency. This threshold also provides a practical basis for comparing variability across both densely spaced inner-city stations and more widely distributed peripheral areas. Across the phases, approximately 73 to 78 per cent of stops recorded average TTV below this one-minute threshold, indicating broad but uneven reliability. These findings are supported by line-specific visualisations presented in the Appendix section Travel Time Volatility, which reveal directional and spatial variation not evident in summary statistics. Patterns of volatility were not uniformly distributed, with each line exhibiting distinct spatial and directional characteristics that shaped the overall stability of service.

    On the Green Line, TTV remained relatively low and spatially contained throughout the data duration. During pre-COVID, average volatility stayed below one minute for over three-quarters of stops, with moderate peaks up to 4.28 minutes appearing around core city-centre locations such as Harcourt and Charlemont during peak hours. Lockdown introduced more directional instability, particularly in outbound services south of Sandyford, where extended segment lengths and lower frequencies likely contributed to increased variability. Visualisations show this volatility remained geographically localised, with little evidence of systemic disruption. Recovery and post-COVID phases marked a return to broader consistency; average TTV values stabilised, with 74 per cent of stops remaining below the one-minute threshold during recovery and 73 per cent post-COVID. Small, residual fluctuations persisted in transitional suburban zones, such as Milltown, but no strong directional asymmetries were observed. This consistent pattern suggests that the Green Line adapted well to pandemic-induced shifts in ridership and maintained operational balance throughout.

    In contrast, the Red Line exhibited higher and more spatially dispersed volatility throughout the same phases. The pre-COVID phase, at inner-city segments such as James’s, Abbey Street, and Heuston, recorded moderate TTV levels, often associated with multimodal interfaces and higher passenger turnover. During the lockdown, inbound volatility peaked at Kylemore and Connolly, while outbound variability was dispersed across western zones, including Tallaght, Cookstown, and Suir Road. The Red Line also recorded the highest segment-level TTV overall, reaching 4.94 minutes. Although 75 per cent of stops remained below the one-minute threshold, this reflects the line’s length, stop density, and directional complexity rather than a uniformly reliable experience. Post-lockdown recovery did not significantly reduce this dispersion. Visual analysis of post-COVID maps shows elevated TTV at outer stops such as Hospital and The Point, with central segments maintaining better control. The persistence of directionally imbalanced variability, particularly during evening outbound service, points to underlying constraints in scheduling resilience and corridor complexity, especially where the line intersects with mixed traffic environments or less predictable demand flows.

    These patterns underscore the importance of spatial and directional sensitivity when interpreting TTV within a high-frequency light rail system. While the majority of stops on both lines met the sub-minute benchmark across all phases, TTV was neither evenly distributed nor without operational consequence. The Green Line demonstrated greater overall stability, with elevated values largely confined to predictable peak-hour segments and minimal directional asymmetry. In contrast, the Red Line exhibited structurally embedded inconsistencies, particularly in outer and interface zones, with volatility persisting even during the time characterised as service recovery. From a passenger perspective, these fluctuations translate into uneven transit predictability, especially for those travelling through complex or peripheral sections of the network. As such, travel time volatility not only reflects short-term operational irregularities but also exposes deeper challenges in sustaining equitable and reliable service delivery across a spatially diverse urban transit system.

\phantomsection
\subsection*{Journey Duration}
\addcontentsline{toc}{subsection}{Journey Duration}

    Although the prior metrics assess service rhythm and internal consistency, journey duration captures the cumulative outcome of these dynamics across the full length of a route. As a composite measure of operational efficiency, it reflects how effectively the network delivers on expected travel times for end-to-end or long-distance riders. TFI provides informal benchmarks of approximately 50 minutes for the Red Line and 40 minutes for the Green Line \parencite{tii_operations}. Across all pandemic phases, the LUAS schedules generally conformed to these expectations, with Red Line averages ranging from 48.4 to 49.7 minutes and Green Line durations slightly more variable, spanning 48.3 to 49.7 minutes. While these times suggest broad operational stability, they obscure important temporal and directional variation that directly shapes perceived reliability. To examine these nuances, journey durations were disaggregated by morning, midday, evening, and night intervals. Results are visualised in the Appendix section Journey Duration, which includes both summary statistics and hour-specific plots. Ultimately, semi-predictable journey durations, especially during peak periods, are a foundational element of user trust and confidence in public transport.

    Morning and midday services demonstrate the most consistent outcomes, with the Red Line exhibiting stable operations. Inbound services during these stretches averaged around 49 minutes, with standard deviations under four minutes, indicating predictable travel during core commuter windows. Outbound trips followed a similar pattern, showing no major directional imbalance. The Green Line, while broadly aligned with expected durations, displayed greater variability, particularly in inbound morning routes, where standard deviations frequently exceeded six minutes. Midday services also retained residual volatility; although outbound durations occasionally dipped below 48 minutes, variability remained elevated. These fluctuations likely reflect the line’s longer inter-stop distances south of Sandyford and less uniform midday demand, which may have challenged frequency regularity and influenced total journey time. While both lines generally delivered expected durations during high-demand hours, the Green Line’s higher variation highlights the importance of spatial and directional sensitivity when interpreting service reliability. Inconsistency can erode user confidence and complicate commuter planning even where averages fall within benchmark ranges.

    Evening and nighttime periods revealed more pronounced discrepancies, with performance diverging by line and direction. The Red Line remained comparatively stable, averaging around 49 minutes across the phases, with standard deviations consistently below four minutes. This suggests effective schedule adherence and minimal late-day disruption, reinforcing the line’s overall operational resilience. The Green Line, by contrast, exhibited more erratic patterns. Evening outbound journeys frequently exceeded 48.5 minutes, with standard deviations rising above seven minutes in some phases, pointing to delays caused by operation gaps, extended dwell times, or congestion in suburban segments. Nighttime services further amplified this inconsistency; outbound durations regularly approached or exceeded 56 minutes, while inbound trips, though slightly shorter, remained variable. These trends reflect the compound effects of reduced frequency, fluctuating boarding times, and directional imbalances introduced by asymmetric routing and suburban branching. For passengers, such fluctuations compromise confidence in end-of-day travel consistency. While mean durations post-COVID remained broadly aligned with pre-COVID expectations, elevated nighttime variability signals persistent challenges in off-peak service planning and resource distribution.

    Journey duration patterns reinforce the broader reliability dynamics observed across the LUAS network, illustrating how time of day shapes both operational outcomes and passenger expectations. Morning and midday services generally demonstrated stability, particularly on the Red Line, reflecting effective control during high-demand commuting windows. In contrast, evening and nighttime periods exposed greater operational strain, most notably on the Green Line, where durations became more prolonged and variable. These discrepancies reveal how reliability is not solely a peak-hour concern but a system-wide challenge that spans the full service day. From a passenger standpoint, prolonged or inconsistent travel can erode confidence in the system, particularly when users rely on predictable services for routine planning. By capturing the cumulative effect of network rhythm, frequency, and directional imbalance, journey duration serves as a critical indicator of both system performance and perceived dependability. Ensuring consistent outcomes across all time periods remains key to maintaining a transit network that is not only operationally sound but also dependable from the perspective of daily riders.


