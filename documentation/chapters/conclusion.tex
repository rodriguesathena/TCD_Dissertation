\section*{Conclusion}
\addcontentsline{toc}{section}{Conclusion}

    This study explored the reliability of Dublin’s main transit network by reconstructing patterns from public AVLS forecasts. Working within a data-limited environment, it developed a replicable framework to infer LUAS journeys and evaluate system efficacy using behaviourally grounded metrics suited to high-frequency operations. Applied across distinct phases of pandemic disruption, the analysis processed over 350,000 reconstructed journeys, revealing persistent gaps in spatial coverage, directional balance, and operational consistency, offering new insight into where and how reliability breaks down.

    In doing so, the research contributes both a methodological template and a diagnostic toolset for performance evaluation under real-world constraints. It demonstrates that even fragmented public forecast data can support scalable, nuanced assessments of regularity, revealing not just whether services run, but how consistently and equitably they do so. More broadly, the study positions reliability as a socially embedded outcome shaped by planning assumptions, operational practices, and the lived realities of passengers navigating uneven operations.

    By making irregularity visible and actionable, a foundation for more adaptive, inclusive, and accountable transit is created; consistency is recognised not as a technical ideal, but as a public responsibility central to mobility justice. Urban transit planning must prioritise equity-driven consistency as a core standard, not a peripheral goal. Highlighting both the limits and possibilities of open data reinforces the value of transparency and adaptability in shaping the future of public transport.

