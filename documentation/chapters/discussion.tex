\section*{Discussion}
\addcontentsline{toc}{section}{Discussion}

\phantomsection
\subsection*{The Pandemic and Network Strain}
\addcontentsline{toc}{subsection}{The Pandemic and Network Strain}

    The COVID-19 pandemic placed sustained operational and behavioural pressure on the LUAS network; however, transit offerings were not completely shut down, instead running at reduced capacity, even during Ireland’s strictest lockdowns \parencite{nta2021report}. Forecast-based reconstruction confirms this continuity, with the number of inferred journeys remaining relatively stable across the duration, averaging around 10,000 journeys per month across both lines, even during the extended lockdown phase. However, sustaining transport levels did not equate to sustaining reliability, as key performance metrics were visibly affected. Service became more uneven at the network’s spatial edges, directional imbalances grew more pronounced, and volatility clustered in central interchange zones. While not part of the dataset, it is notable to highlight that passenger volumes dropped significantly during the early stages and recovered unevenly, with LUAS ridership falling by nearly 60 per cent during the 2020 lockdown alone \parencite{nta2021report}. These behavioural shifts likely contributed to changing demand patterns, complicating service planning and reducing the predictability needed to maintain consistent spacing and frequency across the network. More importantly, these disruptions were not confined to the height of the outbreak as they extended through the recovery phase and into post-COVID operations, indicating that short-term adaptations may have contributed to more persistent reliability gaps. In this context, the pandemic revealed not just temporary disruption but deeper structural limits in the LUAS’s ability to maintain balanced, high-frequency service across a shifting urban landscape.

    While patterns in service reliability evolved across the timeline, these occurred unevenly and with lasting effects. During the pre-COVID phase, performance was relatively stable, with consistent journey durations and fewer extreme headways. However, as the system moved into the lockdown phase, headway regularity declined, particularly at the network’s outer edges, while travel time volatility increased in several central and high-traffic segments. These changes likely reflected both operational constraints and shifting travel behaviour, as travel options were reduced and ridership became more fragmented. The recovery phase saw only limited improvement; while some central corridors regained rhythm, outbound and late-day services remained inconsistent. By the post-COVID phase, journey volumes had returned to near pre-pandemic levels, but performance indicators suggested that many emergency-era adaptations had become structurally embedded. Extended headways, directional imbalance, and concentrated volatility, especially near interchange zones, remained defining features. Rather than reverting to earlier norms, the network appeared to settle into a new baseline of uneven reliability shaped by both pre-existing limitations and the residual effects of crisis-driven adjustments.

    Rather than functioning solely as a disruption, the pandemic exposed deeper structural vulnerabilities in the LUAS network and reshaped its operational norms. While continued service during lockdown demonstrated baseline resilience, the persistence of disrupted patterns after the crisis ended suggests that many emergency adaptations became embedded defaults. The LUAS did not revert to prior norms but instead settled into a new operational rhythm defined by widened headways, uneven spacing, and reliability gaps at the periphery. These shifts highlight the need for more adaptive planning strategies that integrate spatial equity and evolving demand patterns into service design.

\phantomsection
\subsection*{Spatial and Directional Equity}
\addcontentsline{toc}{subsection}{Spatial and Directional Equity}

    Overall performance reveals a structural disconnect between the geographic coverage and the consistent LUAS service required to make coverage truly functional. While central corridors demonstrated relative operational stability, the network’s outer areas consistently experienced extended headways and greater travel-time variability. These patterns were not tied solely to moments of disruption, instead persisting across all phases and indicating that marginal zones are structurally more vulnerable to unreliable operations. In these areas, reliable transit often exists on paper but fails to deliver predictable or dependable mobility in practice, further undermining the infrastructure’s ability to meet everyday travel needs, particularly for passengers without viable alternatives or flexible schedules. Studies have shown that transit networks which expand their geographic reach without corresponding investments in reliability tend to see reduced uptake and wider spatial disparities in access outcomes \parencite{santos2013externalities}. For riders at or near the edge, reliability is not just an added layer of service quality; instead, it is the baseline condition for transit to function as a meaningful option. Addressing spatial equity in high-frequency systems thus requires deliberate planning to ensure that consistency extends throughout the network, not just where demand is densest. These spatial imbalances are further compounded by directional asymmetries that reflect deeper structural assumptions in design.

    Alongside spatial disparities, the LUAS exhibited persistent directional imbalances that further complicate reliability. Inbound services during morning peaks and within central corridors tended to operate with greater consistency, while outbound and late-day services experienced more frequent irregularities, including extended headways and inconsistent timings. These asymmetries suggest that the system remains strongly oriented toward traditional, peak-focused commuting patterns, with less emphasis placed on supporting bidirectional travel throughout the day. Such planning structures overlook the increasing complexity of urban transit, where journeys are not as predictably timed or centrally focused. When outbound services underperform, it disproportionately affects those with non-traditional work hours, caring responsibilities, or limited flexibility in their daily schedules. Research shows that gaps in low-demand directions are often overlooked in frequency planning models, even when those offerings remain essential for transport-dependent users \parencite{jiang2022reliability}. Moreover, equity in transit must account for not just where and when service is available, but whether it performs consistently for a diverse range of travel needs \parencite{martens2016justice}. Addressing directional equity, therefore, requires a shift away from peak-centric logic toward a more distributed model that treats service regularity in all directions as essential to fair and functional network design.

\phantomsection
\subsection*{Methodological Reflection}
\addcontentsline{toc}{subsection}{Methodological Reflection}

    Persistent reliability gaps in LUAS peripheral and outbound segments suggest the continued influence of outdated planning assumptions and reactive models, despite overall network stability. This study adopted a custom methodology tailored to a constrained data environment, where vehicle-level tracking and operational arrival records were unavailable. This approach aligns with established practices in transit research, where evolving prediction streams are used to approximate real-time stop events \parencites{muller2001trip}{sun2016smart}. Furthermore, the selected metrics were grounded in their demonstrated relevance for on-demand transit, particularly where schedule adherence is a weak proxy for service quality \parencites{vanoort2011service}{tirachini2022headway}. As such, the method provided a behaviourally meaningful, replicable framework for assessing system reliability using only publicly accessible data.

    That said, the methodology has notable limitations, most significantly, its exclusion of key external factors known to affect service reliability. Variables such as traffic interference, adverse weather, passenger load, and signal-related delays were not included in the analysis. Some of these elements are limited in accessibility through public sources and, if available, are often inconsistently formatted or distributed across disconnected systems, making integration complex and technically demanding. Accounting for them would have required a fundamentally different analytical approach, likely involving simulation models, multi-source fusion, or machine learning techniques, methods that fall outside the scope and intent of this study. The aim here was not to model causality behind disruptions, but to identify observable reliability patterns using a transparent, replicable framework suitable for constrained contexts. While these omissions limit the ability to explain why performance breaks down, they do not preclude the study from identifying where and how often such breakdowns occur, critical first steps in diagnosing systemic issues. In addition, the dataset used extends only through October 2022, offering a retrospective snapshot of LUAS journeys during and immediately after the COVID-19 pandemic. The findings should therefore be interpreted as reflective of that specific period, with an understanding of potential future changes implemented or necessary.

    Even within these bounds, the methodology proved effective at identifying persistent patterns of irregularity and imbalance in LUAS operations. It demonstrates the value of forecast-based inference for system-level evaluation, particularly in high-frequency or AVLS-driven environments where direct observations are not available. By focusing on observable outcomes, this approach shifts attention towards how service is experienced, supporting a more experience-oriented view of reliability. This, in turn, enables agencies to identify persistent weaknesses and adapt planning to better reflect lived transit conditions. As data transparency and real-time monitoring tools become more common, such frameworks can play a critical role in making network performance more accountable, accessible, and aligned with the goals of equitable provision.

\phantomsection
\subsection*{Implications and Future Research}
\addcontentsline{toc}{subsection}{Implications and Future Research}

    The LUAS network’s ongoing irregularities in peripheral areas and outbound flows reflect a system still shaped by outdated design assumptions and reactive service management. These findings suggest the need for more spatially and temporally responsive planning, moving beyond broad frequency targets or peak-hour optimisation toward strategies attuned to localised demand and lived travel conditions. For agencies like Transport for Ireland, ensuring service dependability in a high-frequency context means not only running routes often, but doing so with consistency and predictability across the entire network. Metrics such as headway regularity and travel time volatility provide more sensitive diagnostics of operations degradation, particularly in areas where reliability is most critical for transport-dependent persons. Monitoring known volatility hotspots, such as interchange zones and line termini, can support more adaptive, evidence-led interventions to strengthen network resilience.

    These results also call for a stronger equity lens in reliability planning. Persistent quality gaps at the edges of the network or during off-peak periods are not merely operational inconsistencies; they are indicators of deeper structural disparities in access and mobility. Such gaps disproportionately affect those whose travel needs often fall outside the design assumptions of peak-oriented, centre-focused service models \parencites{caulfield2021trinity}{hynes2020utility}. Addressing these inequities requires performance frameworks that move beyond aggregate frequency measures and instead prioritise the consistency and predictability of provisions where it is most socially necessary. This means aligning reliability targets with principles of spatial and directional fairness, and ensuring that extended coverage translates into dependable mobility for all users, regardless of geography or time of travel. As urban travel patterns grow more decentralised and trip-making more diverse, transit planning must adapt by embracing flexibility, accessibility, and responsiveness as core tenets of system functionality, not optional enhancements.

    Looking forward, future research could extend this framework in several directions. Integrating operational variables such as passenger loads, signal delays, or disruption alerts would clarify the mechanisms behind inconsistency and allow for richer causal interpretation. Access to more detailed datasets would also enable the use of advanced modelling techniques, whether to simulate resilience under stress or to test interventions. Just as important is the inclusion of rider perspectives: understanding how passengers experience and respond to unreliable service would ground diagnostic findings in lived experience. Combining behavioural insight with network analysis would support strategies that are both operationally effective and socially responsive. In parallel, this type of rule-based forecast reconstruction could be integrated into Transport for Ireland’s internal monitoring systems as a supplementary tool, helping to flag emerging reliability gaps. More immediately, metrics such as headway regularity or volatility could inform practical interventions such as transit signal priority, frequency buffers, or automated dispatch corrections at known bottlenecks. The methodological foundations laid by this study offer a pathway toward more inclusive and adaptive approaches to public transport planning, both within LUAS and beyond.