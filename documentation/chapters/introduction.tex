\section*{Introduction}
\addcontentsline{toc}{section}{Introduction}

As urbanisation continues to spread upwards and outwards, stable public transport connections between dense city centres and peripheral areas are essential to support efficient and sustainable mobility. The LUAS (Irish for ‘speed’) light rail system plays a key role in supporting County Dublin’s ongoing urban sprawl and population growth, offering high-capacity travel across two directional routes. However, like many transit systems, the LUAS faces persistent operational challenges such as peak-hour bunching, inconsistent tram spacing, and schedule deviations. These issues can reduce service reliability, erode passenger trust, and ultimately deter regular ridership. This study evaluates the reliability of LUAS services between 2020 and 2022 by reconstructing tram journeys from historical forecast data and deriving operational performance metrics. In doing so, this research aims to determine whether widely accepted transit indicators can be extracted from publicly available data, and, if so, how they capture real-world service consistency and stability.

Good service reliability is often perceived as a measure of network quality; when absent, it contributes to increased passenger wait-time anxiety and overcrowding. Using Automatic Vehicle Location System (AVLS) data, this study relies on predicted tram movement to understand the performance of the LUAS infrastructure through an unprecedented stress test, the COVID-19 pandemic. This temporal scope provides a natural division of data into four distinct phases to evaluate changes in public transport. Defined as pre-COVID, lockdown, recovery, and post-COVID, the four phases represent a unique operational context that is shaped by changing demand, altered service patterns, and evolving public health restrictions. By comparing key transit reliability metrics such as headway regularity, travel time volatility, and journey duration across these intervals, the system’s adaptability to fluctuating pressures can be investigated. Together, these provide a multifaceted view of performance, capturing both internal and passenger-perceived reliability. The combined analytical approach enables a comprehensive evaluation of the network’s ability to maintain dependable service under rapidly shifting constraints.

As a relatively modern light rail system with limited publicly data, Dublin's LUAS offers a rare opportunity to assess whether meaningful operational insights can be drawn from forecast data alone. This research assesses the possibility of reliably inferring transit metrics from created journeys, providing insight into operational behaviour under conditions of system-wide disruption. As a result, this study contributes to demonstrating the effectiveness and transparency of rule-based approaches for transport analysis using forecast-only data.
